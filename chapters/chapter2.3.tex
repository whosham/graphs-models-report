\subsection{HIL\_STATE 90} 
Sent from simulation to autopilot. This packet is useful for high throughput applications such as hardware in the loop simulations.\\

{\rowcolors{2}{yellow}{yellow}
\centering
\begin{tabular}{ |p{4cm  } |p{7cm} | p{2cm}|m{5em}|}
\hline
Mavlink Message Field(Name,type)&Corresponding MSP Message(Name,id,data,type)& Compatibility & Notes\\
\hline
time\_usec,u64 & Not Found & No & - \\
\hline
\rowcolor{lightgray}
roll,f & MSP\_HIL\_STATE,-,ROLL,u16 & Partially & Mavlink f MSP u16\\
\hline
\rowcolor{lightgray}
pitch,f & MSP\_HIL\_STATE,-,PITCH,u16& Partially & Mavlink f MSP u16 \\
\hline
\rowcolor{lightgray}
yaw,f&  MSP\_HIL\_STATE,-,YAW,u16& Partially & Mavlink f MSP u16 \\
\hline
rollspeed,f & Not Found& No & - \\
\hline
pitchspeed,f & Not Found& No & - \\
\hline
yawspeed,f & Not Found& No & - \\
\hline
lat,i32 & Not Found& No & - \\
\hline
lon,i32 & Not Found& No & - \\
\hline
alt,i32 & Not Found& No & - \\
\hline
vx,i16 & Not Found& No & - \\
\hline
vy,i16 & Not Found& No & - \\
\hline
vz,i16 & Not Found& No & - \\
\hline
xacc,i16 & Not Found& No & - \\
\hline
yacc,i16 & Not Found& No & - \\
\hline
zacc,i16 & Not Found& No & - \\
\end{tabular}
}

\cleardoublepage



\subsection{OPTICAL\_FLOW 100} 
Optical flow from a flow sensor (e.g. optical mouse sensor)\\

{\rowcolors{2}{yellow}{yellow}
\centering
\begin{tabular}{ |p{4cm  } |p{7cm} | p{2cm}|m{5em}|}
\hline
Mavlink Message Field(Name,type)&Corresponding MSP Message(Name,id,data,type)& Compatibility & Notes\\
\hline
time\_usec,u64 & Not Found & No & - \\
\hline
sensor\_id,u8 & Not Found & No & -\\
\hline
\rowcolor{lightgray}
flow\_x,i16 & -,-,flowRateX,u16 & Partially & Mavlink i16 and MSP u16 \\
\hline
\rowcolor{lightgray}
flow\_y,i16&  -,-,flowRateY,u16& Partially & Mavlink i16 and MSP u16 \\
\hline
flow\_comp\_m\_x,f & Not Found & No & -\\
\hline
flow\_comp\_m\_x,f  & Not Found & No & - \\
\hline
quality,u8 &Not Found& No & - \\
\hline
ground\_distance,f & Not Found & No & - \\
\hline
flow\_rate\_x,f & Not Found& No & - \\
\hline
flow\_rate\_y,f & Not Found& No & - \\



\end{tabular}
}

\cleardoublepage




\subsection{RADIO\_STATUS 109} 
Status generated by radio and injected into MAVLink stream.\\

{\rowcolors{2}{yellow}{yellow}
\centering
\begin{tabular}{ |p{4cm  } |p{7cm} | p{2cm}|m{5em}|}
\hline
Mavlink Message Field(Name,type)&Corresponding MSP Message(Name,id,data,type)& Compatibility & Notes\\
\hline
\rowcolor{lightgray}
rssi,u8 & -,-,localrssi,uchar & Partially & Mavlink u8 and uchar MSP \\
\hline
\rowcolor{lightgray}
remrssi,u8 & -,-,remrssi,uchar & Partially & Mavlink u8 and uchar MSP \\
\hline
\rowcolor{lightgray}
txbuf,u8 & -,-,txbuf,uchar & Partially & Mavlink u8 and uchar MSP \\
\hline
\rowcolor{lightgray}
noise,u8& -,-,noise,uchar& Partially & Mavlink u8 and uchar MSP \\
\hline
\rowcolor{lightgray}
remnoise,u8 & -,-,remnoise,uchar& Partially & Mavlink u8 and uchar MSP \\
\hline
\rowcolor{green}
rxerrors,u16 & -,-,rxerrors,u16 & Yes & - \\
\hline
\rowcolor{green}
fixed,u16 & -,-,fixed\_errors,u16& Yes & - \\
\end{tabular}
}

\cleardoublepage


\subsection{SERIAL\_CONTROL 126 } 
Control a serial port. This can be used for raw access to an onboard serial peripheral such as a GPS or telemetry radio. It is designed to make it possible to update the devices firmware via MAVLink messages or change the devices settings. A message with zero bytes can be used to change just the baudrate.\\

{\rowcolors{2}{yellow}{yellow}
\centering
\begin{tabular}{ |p{4cm  } |p{7cm} | p{2cm}|m{5em}|}
\hline
Mavlink Message Field(Name,type)&Corresponding MSP Message(Name,id,data,type)& Compatibility & Notes\\
\hline
device,u8 & Not Found & No & - \\
\hline
flags,u8 & Not Found & No & - \\
\hline
timeout,i16& Not Found & No & - \\
\hline
\rowcolor{lightgray}
baudrate,u32 & MSP\_CF\_SERIAL\_CONFIG,-,msp\_baudrateIndex,u8 & Partially & Mavlink u32 MSP u8 \\
\hline
count,u8 & Not Found & No & - \\
\hline
data,u[8;70] & Not Found & No & - \\
\end{tabular}
}

\cleardoublepage


\subsection{HOME\_POSITION 242 } 
This message can be requested by sending the MAV\_CMD\_GET\_HOME\_POSITION command. The position the system will return to and land on. The position is set automatically by the system during the takeoff in case it was not explicitly set by the operator before or after. The position the system will return to and land on. The global and local positions encode the position in the respective coordinate frames, while the q parameter encodes the orientation of the surface. Under normal conditions it describes the heading and terrain slope, which can be used by the aircraft to adjust the approach. The approach 3D vector describes the point to which the system should fly in normal flight mode and then perform a landing sequence along the vector.\\

{\rowcolors{2}{yellow}{yellow}
\centering
\begin{tabular}{ |p{4cm  } |p{7cm} | p{2cm}|m{5em}|}
\hline
Mavlink Message Field(Name,type)&Corresponding MSP Message(Name,id,data,type)& Compatibility & Notes\\
\hline
\rowcolor{green}
latitiude,i32 & MSP\_WP,118,lat,i32& Yes & - \\
\hline
\rowcolor{green}
longitude,i32 & MSP\_WP,118,lon,i32 & Yes & - \\
\hline
\rowcolor{green}
altitude,i32 & MSP\_WP,118,altitude,i32 & Yes & - \\
\hline
x,f& Not Found & No & - \\
\hline
y,f& Not Found & No & - \\
\hline
z,f & Not Found & No & - \\
\hline
q,f[4] & Not Found & No & - \\
\hline
approach\_x,f & Not Found & No & - \\
\hline
approach\_y,f & Not Found & No & - \\
\hline
approach\_z,f & Not Found & No & - \\
\hline
time\_usec,u64 & Not Found & No & - \\
\end{tabular}
}

\cleardoublepage


\subsection{SET\_HOME\_POSITION 243 } 
The position the system will return to and land on. The position is set automatically by the system during the takeoff in case it was not explicitly set by the operator before or after. The global and local positions encode the position in the respective coordinate frames, while the q parameter encodes the orientation of the surface. Under normal conditions it describes the heading and terrain slope, which can be used by the aircraft to adjust the approach. The approach 3D vector describes the point to which the system should fly in normal flight mode and then perform a landing sequence along the vector.\\

{\rowcolors{2}{yellow}{yellow}
\centering
\begin{tabular}{ |p{4cm  } |p{7cm} | p{2cm}|m{5em}|}
\hline
Mavlink Message Field(Name,type)&Corresponding MSP Message(Name,id,data,type)& Compatibility & Notes\\
\hline
\rowcolor{green}
latitiude,i32 & MSP\_SET\_WP,209,lat,i32& Yes & - \\
\hline
\rowcolor{green}
longitude,i32 & MSP\_SET\_WP,209,lon,i32 & Yes & - \\
\hline
\rowcolor{green}
altitude,i32 & MSP\_SET\_WP,209,altitude,i32 & Yes & - \\
\hline
x,f& Not Found & No & - \\
\hline
y,f& Not Found & No & - \\
\hline
z,f & Not Found & No & - \\
\hline
q,f[4] & Not Found & No & - \\
\hline
approach\_x,f & Not Found & No & - \\
\hline
approach\_y,f & Not Found & No & - \\
\hline
approach\_z,f & Not Found & No & - \\
\hline
time\_usec,u64 & Not Found & No & - \\
\end{tabular}
}

\cleardoublepage




\subsection{DEBUG 254 } 
Send a debug value. The index is used to discriminate between values. These values show up in the plot of QGroundControl as DEBUG N.\\

{\rowcolors{2}{yellow}{yellow}
\centering
\begin{tabular}{ |p{4cm  } |p{7cm} | p{2cm}|m{5em}|}
\hline
Mavlink Message Field(Name,type)&Corresponding MSP Message(Name,id,data,type)& Compatibility & Notes\\

\hline
time\_boot\_ms,u32 & Not Found & No & - \\
\hline
ind,u8 & Not Found & No & - \\
\hline
\rowcolor{lightgray}
value,f & MSP\_DEBUG,-,debug,u32 & Partially & Mavlink f MSP u32 \\
\end{tabular}
}

\cleardoublepage


\subsection{MAV\_CMD\_NAV\_WAYPOINT 16} 
Navigate to waypoint.\\

{\rowcolors{2}{yellow}{yellow}
\centering
\begin{tabular}{ |p{4cm  } |p{7cm} | p{2cm}|m{5em}|}
\hline
Mavlink Message Field(Name,type)&Corresponding MSP Message(Name,id,data,type)& Compatibility & Notes\\
\hline
Hold,undefined & Not found & No & - \\
\hline
Pass Radius,undefined & Not found & No & - \\
\hline
Accept Radius,undefined & Not found & No & - \\
\hline
Yaw,undefined & Not found & No & - \\
\hline
\rowcolor{lightgray}
Latitude,undefined & MSP\_SET\_WP,209,lat,u32& Partially & - \\
\hline
\rowcolor{lightgray}
Longitude,undefined & MSP\_SET\_WP,209,long,u32& Partially & - \\
\hline
\rowcolor{lightgray}
altitude,undefined & MSP\_SET\_WP,209,Althold,u32& Partially & - \\
\end{tabular}
}

\cleardoublepage

